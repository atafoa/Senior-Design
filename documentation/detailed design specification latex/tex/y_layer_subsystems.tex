\subsection{Layer Hardware}

Software running on a Raspberry pi 0w with Raspbian Buster / 2020-02-05
The raspberry pi 0w is small, made in large scale, and is only \$10

\subsection{Layer Operating System}
Raspbian Buster / 2020-02-05 current version of raspbian available

\subsection{Layer Software Dependencies}
Django 3.0 web framework
Cronjobs  

\subsection{Bluetooth communication}
Communicate request to the shutters over bluetooth using python the bluetool library

\begin{figure}[h!]
	\centering
 	\includegraphics[width=0.60\textwidth]{images/hub}
 \caption{Hub subsystem}
\end{figure}

\subsubsection{Subsystem Hardware}
Raspberry pi 0W and ESP32

\subsubsection{Subsystem Operating System}
Raspbian Buster / 2020-02-05 the current version of raspbian available

\subsubsection{Subsystem Software Dependencies}
Python Bluetool library

\subsubsection{Subsystem Programming Languages}
Using Python 2.7 to communicate to the shutter

\subsubsection{Subsystem Data Structures}
The hub will receive HTTP user requests and send the commands to the shutters via Bluetooth, then receive acknowledgments from the shutter and send the acknowledgment to the user. 

\subsubsection{Subsystem Data Processing}
Send commands to shutter through Bluetooth using the bluetool python library

\subsection{Wi-Fi communication}
Receiving HTTP request and sending JSON back to the application

\subsubsection{Subsystem Hardware}
Using a Raspberry Pi 0W

\subsubsection{Subsystem Operating System}
Raspbian Buster / 2020-02-05 the current version of raspbian available

\subsubsection{Subsystem Software Dependencies}
Django 3.0 Web framework
Bluetool


\subsubsection{Subsystem Programming Languages}
System will be built on the current version of Python 3 and Python 2.7

\subsubsection{Subsystem Data Structures}
Using a dictionary to contain the mac address as the key and the name as the value of the shutter
Sending data and receiving request from mobile users through HTTP request

\subsubsection{Subsystem Data Processing}
Sending information to mobile applications in JSON format as a HTTP response to application api calls.

